\begin{abstract}
This thesis employs X-ray scattering to study the structure of two different stacked
lipid membrane systems.
The first part reports the effect on lipid bilayers of the Tat peptide
Y$_{47}$GRKKRRQRRR$_{57}$ from the HIV-1 virus transactivator of translation (Tat) protein.
Synergistic use of low angle X-ray scattering (LAXS) and atomistic molecular dynamics (MD)
simulations indicated Tat peptide binding to neutral dioleoylphosphatidylcholine (DOPC)
headgroups. This binding induced the nearby lipid phosphate groups to move 3 \AA\
closer to the bilayer center. Many of the Tat arginines were as close to the bilayer 
center as the locally thinned lipid phosphate 
groups. Analysis of LAXS from DOPC, DOPC/dioleoylphosphatidylethanolamine (DOPE),
DOPC/dioleoylphosphatidylserine (DOPS), and a mimic of the nuclear membrane
indicated that the Tat peptide decreased the bilayer bending modulus $K_c$
and increased the area per lipid, possibly facilitating Tat membrane translocation.
Although a mechanism for translocation remains elusive, this study suggests that
Tat translocates from the headgroup region.

The second study presents the structure of the asymmetric ripple phase formed
by dimyristoylphosphatidylcholine. 
We determined the most detailed ripple phase structure by combining synchrotron 
LAXS and wide angle X-ray scattering (WAXS) from highly aligned multilamellar samples.
We derived three intensity corrections to calculate the X-ray form factors from 
the 52 measured reflections.
The LAXS analysis provided a high resolution two-dimensional electron density map.
The ripple major arm was demonstrated to be consistent with the gel phase, 
and the major and minor arm structures were clearly different, supporting the 
coexistence of different molecular organizations.
The minor arm electron density profile was qualitatively consistent
with interdigitated chain packing previously proposed by MD simulations.
Analysis of high resolution near grazing incidence WAXS showed that
major arm hydrocarbon chains were tilted parallel to the ripple 
plane by 18\textdegree\ with respect to the bilayer local normal, 
toward the next nearest neighbor similarly to the gel $L_{\beta F}$ 
rather than the \LbetaI\ phase.
By measuring the Bragg rod lengths in transmission WAXS, we determined that 
major arm chains in opposing leaflets were coupled.
The LAXS and WAXS results together indicated that chains in the major arm
were shorter by 1.3 \AA\ compared to the gel phase, suggesting a
gauche-trans-gauche kink in the ripple major arm.
In contrast to the LAXS analysis, the measured nGIWAXS was consistent with 
disordered chains in the minor arm similarly to the fluid $L_\alpha$ phase.

\end{abstract}
