\chapter{Ripple Phase}
In this chapter, I discuss the chain packing in the DMPC ripple phase. 
I should probably copy a lot of stuff from my old report.

\section{Introduction}

\section{Materials and Methods}
\subsection{Sample Preparation}
DMPC was purchased from Avanti Polar Lipids and used without further purification.
Oriented thin film was deposited following the rock and roll procedure.  
In previous synchtrotron experiments, the samples were created and annealed 
more than a week in advance and stored in a refrigerator. The orientation
quality of these samples were found to be worse than the quality soon after
the samples were annealed. Therefore, to ensure the best sample quality, the 
sample was annealed for approximately 12 hours just before the X-ray experiment.
Figure X compares a sample scattering in 2011 and 2013 synchrotron runs. 
Because of substantial degree of mosaic spread in the 2011 sample, many peaks
overlapped, rendering the accurate measurement of integrated intensity difficult.
Figure XX shows a picture of the annealing chamber. To achieve gentle but
efficient hydration of a sample, filter papers were installed to cover the 
sample. For successful annealing, it must be emphasized that the annealing 
chamber equilibrates in the over prior to putting the samples in the chamber.
When the sample was put in the chamber with its water at room temperature and
then the system was placed inside the oven, warm vapor condensed on the cooler
sample, causing so called flooding of oriented sample. A small drop of water
on the oriented film was deterimental for the orientation because the bilayers
tend to peel off, resulting in entropy-driven formation of unoriented vesicles 
in the water subphase. 

The sample for grazing incident wide angle study was prepared in the same way 
as for low angle study. In order to minimize the geometric broadening, the 
sample was further trimmed down to 1 mm in width.

The sample for transmission study was deposited on a thin, 35 micron, silicon
wafer. Because the wafer was very fragile, attaching the sample to a sticky 
thing was impossible. Instead, the sample was attached to a plastic cap of 
a small vial with a small amount of heat sink compound at a conrner of the 
wafer. The wafer was stable enough for rocking. The sample deposited on a 
glass cover slip (70 microns) was prepared similarly.  

\subsection{Low Angle Diffraction Experiment}
The same setup as described in Tat chapter was used for low angle diffraction
experiment. In order to achieve a D-spacing comparable to that of Wack and Webb
($D \approx 57.9 \AA$), the current to the Peltier was reversed, which heats the 
Peltier surface the sample was situated on. 

The integrated intensity of each peak was obtained by putting a box around a
peak and summing up the intensity in those pixels that fall inside the box.
The background scattering was estimated by measuring the intensity in pixels
near the peak but not containing any peak tail. The choice of box side was 
made according to the width of each peak. Because of mosaic spread in the sameple,
peaks were wider for higher orders. Accordingly, the box was made wider for higher
orders. The box size was chosen so that approximately 90\% of the peak intensity
was counted toward the integrated intensity.

A few peaks in the ripple phase
were very strong, leading to saturation of CCD pixels. A nominally 25 micron 
molybdinum attenuator was inserted in the upstream to reduce the intensity
of the X-ray beam and one second exposure was collected. The integrated intensity 
of the most strong, (1,0) peak was measured from this short exposure. To
measure the actual attenuation factor, a one second exposure without the 
attenuator was also taken. Comparison of these two images yielded an 
attenuation factor of 7.8. The intensity of (2,0) and (2,-1) peaks were also
measured in this one second exposure with an attenuator. All the other peaks
were observed without saturation in 60 second exposure. To properly scale the 
strong orders, (1,0) peak intensity was multipied by 7.8 x 60 and (2,0) and
(2,-1) were multiplied by 60. 

The integrated intensity, peak position in pixel, the size of box, and estimated
background for all the peaks are shown in Appendix.

\subsection{Near Grazing Angle Experiment}
Plot along the peak, along the arc, horizontal, vertical, etc.

\subsection{Transmission Experiment}
No strong order on the equator. Subtraction of water scattering from the
background image.

\subsection{Model}


\section{Results}
\subsection{LAXS}

\subsection{WAXS}

\section{Discussion}
Comparison with previous unoriented/oriented stuff?

Which theories are consistent/inconsistent with the results of this study?


\section{Conclusion}
Well, the ripple phase is the greatest phase in the lipid bilayers. Our detailed
work lead to deeper insight into the formation of this phase. Future experiments
include the high resolution transmission experiment, where both geometric 
broadening and energy dispersion are minimized. The expected resolution 
is the width of the X-ray beam, which is about 3 pixels. This experiment 
doubles the best resolution achieved in this work. 
Another slightly different high resolution experiment is to use silicon 
crystal analyzer downstread of the sample, which completely remove geoemtric
boradening. The downside of this type of high resolution experiment is that
only one point in q-space is probed at any given exposure, so to get a full
2D map of wide angle scattering is time consuming.  
