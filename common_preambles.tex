%%%%%%%%%%%%%%%%% common packages are included here %%%%%%%%%%%%%%%%%
%\usepackage[pdftex,draft]{graphicx}
\usepackage[pdftex]{graphicx}
\usepackage[pdftex]{hyperref}
\usepackage{epstopdf}
\usepackage{tabularx}
\usepackage[lofdepth,lotdepth]{subfig} % subfloat
\usepackage{bm} % bold math
\usepackage{color}
\usepackage[centertags]{amsmath}
\usepackage{mathrsfs}
\usepackage{amsmath}
\usepackage{mathtools}
\usepackage{amssymb}
\usepackage{amsfonts}
\usepackage{amsthm}
\usepackage{newlfont}
\usepackage{textcomp,gensymb} % for \textcelsius, \textdegree, and \degree
\usepackage{syntonly}
\usepackage[toc,page]{appendix}
\usepackage{setspace}
\usepackage[export]{adjustbox}
\usepackage[percent]{overpic}
\usepackage{booktabs}
\usepackage{cite}
\usepackage{notoccite}

%%%%%%%%%%%%%%%%%%%% some declaration %%%%%%%%%%%%%%%%%%%%%
\DeclareGraphicsExtensions{.pdf,.PDF,.png,.PNG,.jpg,.eps}
%\graphicspath{{./figures/}}
\onehalfspacing
%\doublespacing

%%%%%%%%%%%%%%%%%%% new commands are defined here %%%%%%%%%%%%%%%%%%%%%
\newcommand{\dg}{$^{\circ}$} % degree symbol
\newcommand{\iang}{\AA$^{-1}$} % inverse Angstrom symbol
\newcommand{\degC}{$^{\circ}\mathrm{C}$} % degree Celcius
\newcommand{\Eq}[1]{Eq.\,(\ref{#1})} % reference to an equation

% Some mathematical (physical) quantities and symbols that are used often
\newcommand{\xhat}{\mathbf{\hat{x}}}
\newcommand{\yhat}{\mathbf{\hat{y}}}
\newcommand{\zhat}{\mathbf{\hat{z}}}
\newcommand{\kin}{\mathbf{k}_{\mathrm{in}}}
\newcommand{\kout}{\mathbf{k}_{\mathrm{out}}}
\newcommand{\Tat}{\mathrm{Tat}}
\newcommand{\DOPC}{\mathrm{DOPC}}
\newcommand{\cm}{\mathrm{cm}}
\newcommand{\dx}{\mathop{dx}}
\newcommand{\dy}{\mathop{dy}}
\newcommand{\dz}{\mathop{dz}}
\newcommand{\dr}{\mathop{dr}}

% To simplify some formatting issues
\newcommand{\pars}[1]{\mathopen{}\left( #1 \right)\mathclose{}} % () without extra spaces due to \left and \right
\newcommand{\bracks}[1]{\mathopen{}\left[ #1 \right]\mathclose{}} % [] without extra spaces
\newcommand{\angles}[1]{\mathopen{}\left\lange #1 \right\rangle\mathclose{}} % <> without extra spaces
\newcommand{\braces}[1]{\mathopen{}\left\lbrace #1 \right\rbrace\mathclose{}} % {} without extra spaces
\newcommand{\ds}[1]{\displaystyle{#1}}%
\newcommand{\+}{^{\dagger}}%                                
\newcommand{\partiald}[3][]{{\partial^{#1}#2 \over \partial {#3}^{#1}}}%

% Hyphenation
\hyphenation{multi-lamellar}
\hyphenation{table}
