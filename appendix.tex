\chapter{Tat}
\section{Analysis of Fixed Angle Data using NFIT}\label{sec:fixed_angle_analysis}
In this section, I propose a slightly new method to analyze the diffuse 
scattering data. This method may allow us to measure the X-ray form factor
at lower $q_z$ than we have traditionally measured. 

\subsection{Theory}

\subsection{Results}

\section{Proper Incorporation of Mosaic Spread to NFIT analysis}\label{sec:mosaic_spread}
First we describe the theory of mosaic spread for diffuse scattering. 
Next we discuss some simplification. Third, we discuss the program.
Fourth, we show the results.

\subsection{Mosaic Spread: Calculation}
In this section, an analytical framework for a measurement of mosaic spread will 
be developed. Let us imagine that a sample is made up of many small domains 
that are 
tilted from the direction perpendicular to the substrate normal by some amount. 
A "perfect" domain is a domain that is parallel to the substrate plane.
Then, we can consider a probability distribution function, $P(\alpha)$, 
representing a probablity of finding a domain with tilt $\alpha$, which is the 
angle
between the substrate normal and the tilted domain normal. Here, we have
assumed the rotational symmetry about the substrate normal, so that the 
distribution
does not depend on the azimuthal angle, $\beta$. The normalization condition on 
the probability distribution is 
\begin{equation}
  1 = \int_0^{2\pi} \!\! \mathrm{d}\beta  
      \int_0^{\frac{\pi}{2}} \! \mathrm{d}\alpha \, \sin\alpha \, P(\alpha).
\end{equation}
The object of this section is to derive the x-ray scattering structure factor 
including the probability distribtion. The cooridate system employed here is 
such that x, y, and
z-axes of the zero tilt domain, that is, a domain parallel to the substrate, 
coincide with the lab x, y, and z-axes 

First, we want to calculate the structure factor for a domain tilted by 
$\alpha$ and $\beta$, expressed in the lab coordinates (Need a figure). 
For this, we need to express
$\mathbf{q}$ in terms of . We imagine
rotating the coorinates about the y-axis first, and then about the z-axis. In
other words, we apply the appropriate rotation matrices to , y, and 
z-axes. The rotation matrix for rotaing a vector about y-axis is given by
\begin{equation}
  \begin{pmatrix} 
    \cos\alpha & 0 & -\sin\alpha \\ 
    0 & 1 & 0 \\
    \sin\alpha & 0 & \cos\alpha 
  \end{pmatrix}
\end{equation}
and for ratating about z-axis
\begin{equation}
  \begin{pmatrix} 
    \cos\beta & \sin\beta & 0 \\ 
    -\sin\beta & \cos\beta & 0 \\
    0 & 0 & 1 
  \end{pmatrix}
\end{equation}
Then, what we want is
\begin{equation}
  \mathbf{\hat{x}}' = 
  \begin{pmatrix} 
    \cos\beta & \sin\beta & 0 \\ 
    -\sin\beta & \cos\beta & 0 \\
    0 & 0 & 1 
  \end{pmatrix}
  \begin{pmatrix} 
    \cos\alpha & 0 & -\sin\alpha \\ 
    0 & 1 & 0 \\
    \sin\alpha & 0 & \cos\alpha 
  \end{pmatrix}
  \begin{pmatrix}
    1 \\
    0 \\
    0
  \end{pmatrix}
  = 
  \begin{pmatrix}
    \cos\alpha\cos\beta \\
    \cos\alpha\sin\beta \\
    -\sin\alpha
  \end{pmatrix}
\end{equation}
\begin{equation}
  \mathbf{\hat{y}}' = 
  \begin{pmatrix} 
    \cos\beta & \sin\beta & 0 \\ 
    -\sin\beta & \cos\beta & 0 \\
    0 & 0 & 1 
  \end{pmatrix}
  \begin{pmatrix} 
    \cos\alpha & 0 & -\sin\alpha \\ 
    0 & 1 & 0 \\
    \sin\alpha & 0 & \cos\alpha 
  \end{pmatrix}
  \begin{pmatrix}
    0 \\
    1 \\
    0
  \end{pmatrix}
  =
  \begin{pmatrix}
    -\sin\beta \\
    \cos\beta \\
    0
  \end{pmatrix}
\end{equation}
\begin{equation}
  \mathbf{\hat{z}}' = 
  \begin{pmatrix} 
    \cos\beta & \sin\beta & 0 \\ 
    -\sin\beta & \cos\beta & 0 \\
    0 & 0 & 1 
  \end{pmatrix}
  \begin{pmatrix} 
    \cos\alpha & 0 & -\sin\alpha \\ 
    0 & 1 & 0 \\
    \sin\alpha & 0 & \cos\alpha 
  \end{pmatrix}
  \begin{pmatrix}
    0 \\
    0 \\
    1
  \end{pmatrix}
  =
  \begin{pmatrix}
    \sin\alpha\cos\beta \\
    \sin\alpha\sin\beta \\
    \cos\alpha
  \end{pmatrix}
\end{equation}
Then, the components of $\mathbf{q}$ represented in the rotated coodinates, 
denoted 
by $\mathbf{q'}$, are the projection of $\mathbf{q}$ on x$'$, y$'$, and 
z$'$-axes, 
that is,
\begin{equation}
  q_x' = \mathbf{q} \cdot \mathbf{\hat{x}'} 
       = q_x\cos\alpha\cos\beta + q_y\cos\alpha\sin\beta -q_z\sin\alpha  
\end{equation}
\begin{equation}
  q_y' = \mathbf{q} \cdot \mathbf{\hat{y}'} 
       = -q_x\sin\beta + q_y\cos\beta  
\end{equation}
\begin{equation}
  q_z' = \mathbf{q} \cdot \mathbf{\hat{z}'} 
       = q_x\sin\alpha\cos\beta + q_y\sin\alpha\sin\beta + q_z\cos\alpha  
\end{equation}
The transformation rule we are looking for is 
\begin{equation}
  \cos\theta' = \frac{q_z'}{q} 
              = \sin\theta\sin\alpha\cos(\phi-\beta) + \cos\theta\cos\alpha 
  \label{eq:theta'}
\end{equation}
and
\begin{equation}
  \tan\phi' 
    = \frac{q_y'}{q_x'}
    = \frac{\sin\theta\sin(\phi-\beta)}{\sin\theta\cos\alpha\cos(\phi-\beta) 
                                       -\cos\theta\sin\alpha}
  \label{eq:phi'}
\end{equation}
The structure factor of the tilted domain in the lab coordinates is simply 
given 
by $S(\mathbf{q'})=S(q,\theta',\phi')$. Summing over all the domains, we get 
for the total structure factor
\begin{equation}
  S_M(q,\theta,\phi) 
    = \int_0^{2\pi} \!\! \mathrm{d}\beta \int_0^{\frac{\pi}{2}} 
      \mathrm{d}\alpha \, \sin\alpha \, S(q,\theta',\phi') \, P(\alpha)
  \label{eq:SM}
\end{equation}
with Eq.\,(\ref{eq:theta'}) and Eq.\,(\ref{eq:phi'}).

For $\theta=0$, $\theta'=\alpha$ and $\phi'=0$ or $\pi$, so we have
\begin{align}
  S_M(q,0) 
    &\sim \int_0^{\frac{\pi}{2}} \mathrm{d}\alpha \, \sin\alpha \, 
          S(q,\alpha,0) \, P(\alpha)
          + \int_0^{\frac{\pi}{2}} \mathrm{d}\alpha \, \sin\alpha \, 
            S(q,\alpha,\pi) \, P(\alpha) \\
    &= \int_{-\frac{\pi}{2}}^{\frac{\pi}{2}} \mathrm{d}\alpha \, \sin\alpha \, 
       S(q,\alpha) \, P(\alpha)
\end{align} 
if we understand $S(q,\alpha)$ to be the structure factor on the ($q_r$,$q_z$) 
plane. This shows that $S_M(q,0)$ is equal to the convolution of the 
distribution
function and the original structure factor. In general, however, $S_M(q,\theta)$ 
is 
not a convolution of the distribution function and the structure factor.

Given Eqs.\,(\ref{eq:theta'}), (\ref{eq:phi'}), and (\ref{eq:SM}), we want to
 show that
mosaic spread acts as one dimensional convolution in the x-ray structure factor:
\begin{equation}
  S_M(q,\theta) 
    = \int_{-\pi}^{\pi} \mathrm{d}\alpha \, S(q,\theta-\alpha) \, P(\alpha)
  \label{eq:conv}
\end{equation}
The structure factor representing Bragg peaks in the spherical coordinates are 
written as
\begin{equation}
  S(q,\theta,\phi) 
    \sim \frac{\delta(q-\frac{2\pi h}{D})}{q^2} 
         \delta(\cos\theta-1) \delta(\phi)
  \label{eq:Bragg}
\end{equation}
where $\delta(x)$ is the Dirac delta function. Plugging \Eq{eq:Bragg} in 
\Eq{eq:SM},
we obtain $\phi-\beta=0$. Using this condition, we get 
\begin{equation}
  S_M(q,\theta) \sim \frac{\delta(q-\frac{2\pi h}{D})}{q^2}P(\theta)\sin\theta,
\end{equation}
which shows that we can directly measure the probability distribution 
experimentally
by looking at the intensity along $q=2\pi h/D$. In the next section, we will 
discuss 
the relevant experimental techniques.

\subsection{Mosaic Spread: Experiment}
In this section, we discuss experimental procedures to probe appropriate 
$q$-space
to measure the mosaic distribution, $P(\alpha)$. In our setup, the angle of 
incidence between the beam and substrate, denoted by $\omega$, can be varied. A 
conventional method to measure mosaicity distribution is a rocking scan, where
one measures the integrated intensity of a given Bragg peak as a function of 
$\omega$ with a fixed detector position. In a non-conventional method called
ring analysis, one measures the intensity as a function of $\eta$ on a two
dimensional detector. First, we want to show that the two methods mentioned 
above in fact measure the mosaicity disitribution and therefore are equivalent
to each other.

Let $\omega$ be the angle of incidence, $2\theta$ be the total scattering angle,
$p_x$ be the pixel number in the horizontal direction, $p_z$ be the pixel 
number in the vertical direction, $eta$ be the angle measured from the 
$p_z$-axis 
on the detector. $\Delta p$ is 0.07113 mm/pixel. $q_x=\mathbf{q} \cdot 
\hat{\mathbf{x}}$, $q_y=\mathbf{q} \cdot \hat{\mathbf{y}}$, and $q_z=\mathbf{q} 
\cdot \hat{\mathbf{z}}$, 
where $\hat{\mathbf{x}}$, $\hat{\mathbf{y}}$, and $\hat{\mathbf{z}}$ are all 
defined on the sample space. This means that $\hat{\mathbf{y}}$ and 
$\hat{\mathbf{z}}$ both rotate as $\omega$ is varied while $\hat{\mathbf{x}}$ 
is always perpendicular to the beam. What we need is a set of transformation
rules for going from the detector space (pixels) to the sample $q$-space.
With them, we would know how to trace out a line on the detector in order
to measure the mosaicity distribution. 


The incoming and outgoing wavevectors of the x-ray beam in Fig. XXX 
are given by
\begin{equation}
  \kin = \frac{2\pi}{\lambda} \yhat, \quad
  \kout = 
    \frac{2\pi}{\lambda} \left( 
      \sin 2\theta \cos\phi \, \xhat
      + \cos 2\theta \, \yhat
      + \sin 2\theta \sin\phi \, \zhat 
    \right),
  \label{eq:kinkout}
\end{equation}
where $\lambda$ is the wavelength of x-ray. The scattering vector is
the difference between $\kin$ and $\kout$,
\begin{align}
  \mathbf{q} &= \kout - \kin \nonumber \\
             &= q \left( 
                  \cos\theta\cos\phi \, \xhat - \sin\theta \, \yhat
                  + \cos\theta\sin\phi \, \zhat
                \right),
  \label{eq:q_vector}
\end{align}
where $q=4\pi\sin\theta/\lambda$ is the magnitude of the scattering vector. 
When the sample is rotated by $\omega$ about the x-axis in the clockwise 
direction as shown in Fig. XXX, the sample coodinates written in terms of 
the lab coordinates are  
\begin{equation}
  \mathbf{\hat{e}_x} = \xhat, \quad
  \mathbf{\hat{e}_y} = \cos\omega\,\yhat + \sin\omega\,\zhat, \quad
  \mathbf{\hat{e}_z} = -\sin\omega\,\yhat + \cos\omega\,\zhat.
  \label{eq:smp_coord}
\end{equation}
From \Eq{eq:q_vector} and \Eq{eq:smp_coord}, we find the projection of 
$\mathbf{q}$ on the sample coordinates to be
\begin{align}
  q_x &= \mathbf{q}\cdot\mathbf{\hat{e}_x} 
       = q\cos\theta\cos\phi 
       \label{eq:qx} \\
  q_y &= \mathbf{q}\cdot\mathbf{\hat{e}_y} 
       = q\left(-\sin\theta\cos\omega + \cos\theta\sin\phi\sin\omega\right) 
       \label{eq:qy} \\
  q_z &= \mathbf{q}\cdot\mathbf{\hat{e}_z} 
       = q\left(\sin\theta\sin\omega + \cos\theta\sin\phi\cos\omega\right).
       \label{eq:qz}
\end{align}
With respect to the beam, the position on the detector is given by
\begin{equation}
  X = S \tan 2\theta \cos\phi, \quad Z = S \tan 2\theta \sin\phi.
\end{equation} 
The pixels on the detector are directly proportional to $X$ and $Z$. Thus,
these equations define the transformation rules from the detector space
to the sample q-space and vice versa.

In terms of these coordinates, in the rocking scan, $\phi=\pi/2$ and 
$q=2\pi h/D$
while $\omega$ is varied about $\theta_B$, where $\theta_B$ is the Bragg
angle for a Bragg peak that is focused on. Using $q=4\pi\sin\theta/\lambda$, 
we recover the Bragg condition, $2D\sin\theta=h\lambda$. Plugging $\phi=\pi/2$
in Eq.\,(\ref{eq:qx}), (\ref{eq:qy}), and (\ref{eq:qz}), and taking 
$\theta=\theta_B$, 

\subsection{Results}

\section{Domain Size Distribution: Gaussian and Exponential}

\section{Hard Wall Constraints in SDP}

\section{Some More Details of Tat Stuff}

\chapter{Ripple Phase}
\section{Derivation of the contour part of the form factor}
In this section, we derive $\FC$. The ripple profile, $u(x)$ is given by
\begin{equation}
  u(x) = \left\{
    \begin{array}{ccc}
    -\frac{A}{\lambda_r-x_0}\left(x+\frac{\lambda_r}{2}\right) 
      & \text{for} 
      & -\frac{\lambda_r}{2} \leq x < -\frac{x_0}{2} \\
    \frac{A}{x_0}x 
      & \text{for} 
      & -\frac{x_0}{2} \leq x \leq \frac{x_0}{2} \\
    -\frac{A}{\lambda_r-x_0} \left(x-\frac{\lambda_r}{2}\right)
      & \text{for} 
      & \frac{x_0}{2} < x \leq \frac{\lambda_r}{2}
    \end{array} \right.
\end{equation}

The contour part of the form factor is the Fourier transform of the contour
function, $C(x,z)$,
\[
  \FC(\mathbf{q}) = \frac{1}{\lambda_r}
  \int_{-\frac{\lambda_r}{2}}^{\frac{\lambda_r}{2}}\dx
  \int_{-\frac{D}{2}}^\frac{D}{2}\dz 
  C(x,z) e^{iq_zz} e^{iq_xx}
\] 
As discussed in section X, the modulated models allow
the electron density to modulate along the ripple direction, $x$. This means
\begin{align}
  C(x,z) &= \left\{
  \begin{array}{ccc}
    f_1\delta[z-u(x)] & \text{for} & -\frac{\lambda_r}{2} \leq x < -\frac{x_0}{2} \\
    \delta[z-u(x)] & \text{for} & -\frac{x_0}{2} < x < \frac{x_0}{2} \\
    f_1\delta[z-u(x)] & \text{for} & \frac{x_0}{2} \leq x < \frac{\lambda_r}{2} \\    
  \end{array}
  \right. \nonumber\\
  &+ f_2\,\delta\!\pars{x+\frac{x_0}{2}}\delta\!\pars{z+\frac{A}{2}} 
   + f_2\,\delta\!\pars{x-\frac{x_0}{2}}\delta\!\pars{z-\frac{A}{2}}.
\end{align}
The contribution from the minor arm is
\begin{align}
  & \frac{1}{\lambda_r}
  \int_{-\frac{\lambda_r}{2}}^{-\frac{x_0}{2}}\dx e^{iq_xx} e^{iq_zu(x)}
  + \int_{\frac{x_0}{2}}^{\frac{\lambda_r}{2}}\dx e^{iq_xx} e^{iq_zu(x)} \nonumber\\
  &= \frac{1}{\lambda_r}
     \int_{\frac{x_0}{2}}^{\frac{\lambda_r}{2}}\dx 
     e^{-i\left[q_xx-q_z\frac{A}{\lambda_r-x_0}\left(x-\frac{\lambda_r}{2}\right)\right]}
     + \int_{\frac{x_0}{2}}^{\frac{\lambda_r}{2}}\dx 
     e^{i\left[q_xx-q_z\frac{A}{\lambda_r-x_0}\left(x-\frac{\lambda_r}{2}\right)\right]} \nonumber\\
  &= \frac{2}{\lambda_r}
     \int_{\frac{x_0}{2}}^{\frac{\lambda_r}{2}}   
     \cos\bracks{\pars{q_x-q_z\frac{A}{\lambda_r-x_0}}x
                 +q_z\frac{A}{\lambda_r-x_0}\frac{\lambda_r}{2}} \label{eq:minor_arm1}
\end{align}
Using a trigonometric identity, 
\[
  \sin u-\sin v = 2\cos[(u+v)/2]\sin[(u-v)/2],
\]
and defining 
\begin{equation}
  \omega(\mathbf{q}) = \frac{1}{2}\left(q_xx_0 + q_zA\right),
\end{equation}
we further simplify Eq.~(\ref{eq:minor_arm1}),
\begin{align}
  &= \frac{2}{\lambda_r}\frac{\lambda_r-x_0}{\frac{1}{2}q_x\lambda_r - \omega} 
     \cos\bracks{\frac{1}{2}\left(\frac{1}{2}q_x\lambda_r + \omega\right)} 
     \sin\bracks{\frac{1}{2}\left(\frac{1}{2}q_x\lambda_r - \omega\right)} \nonumber\\
  &= \frac{1}{\lambda_r}\frac{\lambda_r-x_0}{\frac{1}{2}q_x\lambda_r - \omega} 
     \cos\bracks{\frac{1}{2}\left(\frac{1}{2}q_x\lambda_r + \omega\right)} 
     \frac{\sin\left(\frac{1}{2}q_x\lambda_r - \omega \right)}
          {\cos\bracks{\frac{1}{2}\left(\frac{1}{2}q_x\lambda_r - \omega\right)}} \nonumber\\
  &= \frac{\lambda_r-x_0}{\lambda_r}
     \frac{\cos\bracks{\frac{1}{2}\left(\frac{1}{2}q_x\lambda_r + \omega\right)}}
          {\cos\bracks{\frac{1}{2}\left(\frac{1}{2}q_x\lambda_r - \omega\right)}}
     \frac{\sin\left(\frac{1}{2}q_x\lambda_r - \omega\right)}
          {\frac{1}{2}q_x\lambda_r - \omega}.
\end{align}
Similarly, we calculate the contribution from the major arm,
\begin{align}
  \frac{1}{\lambda_r}\int_{-\frac{x_0}{2}}^{\frac{x_0}{2}}\dx 
  e^{i\left(\frac{q_zA}{x_0} + q_x \right)x}
  &= \frac{2}{\lambda_r}\int_{0}^{\frac{x_0}{2}}\dx \cos\left(\frac{q_zA}{x_0} + q_x\right)x \nonumber\\ 
  &= \frac{x_0}{\lambda_r}\frac{\sin\omega}{\omega}
\end{align}
The contribution from the kink region is 
\begin{align}
  & \frac{1}{\lambda_r}\iint\dx\dz
  \bracks{\delta\!\pars{x+\frac{x_0}{2}}\delta\!\pars{z+\frac{A}{2}} 
   + \delta\!\pars{x-\frac{x_0}{2}}\delta\!\pars{z-\frac{A}{2}}}
  e^{iq_xx} e^{iq_zz} \nonumber\\
  &= \frac{2}{\lambda_r}\cos\omega.
\end{align}
Therefore,
\begin{align}
  \FC(\mathbf{q}) 
  &= \frac{x_0}{\lambda_r}\frac{\sin\omega}{\omega} + 
  f_1\frac{\lambda_r-x_0}{\lambda_r}
  \frac{\cos\bracks{\frac{1}{2}\left(\frac{1}{2}q_x\lambda_r + \omega\right)}}
       {\cos\bracks{\frac{1}{2}\left(\frac{1}{2}q_x\lambda_r - \omega\right)}}
  \frac{\sin\left(\frac{1}{2}q_x\lambda_r - \omega\right)}
       {\frac{1}{2}q_x\lambda_r - \omega} \nonumber\\
  &+ \frac{2f_2}{\lambda_r}\cos\omega
\end{align}

%%%%%%%%%%%%%%%%%%%%%%%%%%%%%%%%%%%%%%%%%%%%%%%%%%%%%%%%%%%%%%%%%%%%%%%%%%%%%%%
\section{Rotation of a Two-Dimensional Function}
Let us consider rotating a function, $f(x,z)$ in two dimensions by an angle, 
$\psi$, in the counterclockwise direction (see Fig. X). This is easily 
achieved by rotating the coordinate system by $\psi$ in the clockwise direction. 
Let rotated coordinates be $x'$ and $z'$. A point in the original coodinates,
($x$, $z$), is written as ($x'$, $z'$) in the new coordinates. More specifically,
the point P is written as 
$\mathbf{P}=x\xhat+z\zhat=x'\xhat'+z'\zhat'$. $\xhat$ and $\zhat$ in
the $x'z'$ coordinate system are written as 
\begin{align}
  \xhat &= \cos\psi\xhat'+\sin\psi\zhat' \\
  \zhat &= -\sin\psi\xhat'+\cos\psi\zhat'.
\end{align}
Pluggin these in $\mathbf{P}=x\xhat+z\zhat$ leads to
\begin{align}
  x' &= x\cos\psi - z\sin\psi \\
  z' &= z\cos\psi + x\sin\psi,
\end{align}
the inverse of which is
\begin{align}
  x &= x'\cos\psi + z'\sin\psi \\
  z &= -x'\sin\psi + z'\cos\psi.
\end{align}
Using the latter equations, $f(x,z)$ can be expressed in terms of $x'$ and $z'$. 
The resulting function $f(x',z')$ is the rotated version of $f(x,z)$. 

As an 
example, let us consider a Dirac delta function located at $(x,z)=(0,\zh)$,
that is, $f(x,z)=\delta(x)\delta(z-\zh)$. After the rotation by $\psi$, it 
becomes
\begin{align*}
  f(x,z) 
  &\rightarrow 
    \delta(x\cos\psi+z\sin\psi) \delta(-x\sin\psi+z\cos\psi-\zh) \\
  &= \frac{\delta(x+z\tan\psi)}{|\cos\psi|}
     \frac{\delta(-x\sin\psi\cos\psi+z\cos^2\psi-\zh\cos\psi)}{1/|\cos\psi|} \\
  &= \delta(x+z\tan\psi)\delta(z\tan\psi\sin\psi\cos\psi+z\cos^2\psi-\zh\cos\psi) \\
  &= \delta(x+z\tan\psi)\delta(z-\zh\cos\psi),
\end{align*}
which is a part of the expression for $T_\psi(x,z)$ in the simple delta 
function model.

%%%%%%%%%%%%%%%%%%%%%%%%%%%%%%%%%%%%%%%%%%%%%%%%%%%%%%%%%%%%%%%%%%%%%%%%%%%%%%%
\section{Derivation of the transbilayer part of the form factor in the 2G hybrid model}
In this section, we derive the trasbilayer part of the form factor calculated
from the 2G hybrid model discussed in section X.
Defining $z'=-x\sin\psi+z\cos\psi$, the Fourier transform of a Gaussian function 
along the line tilted from $z$-axis by $\psi$ is
\begin{align}
  & \iint\dz\dx \rhoh{i} \exp\braces{-\frac{(z'-\zh{i})^2}{2\sigmah{i}^2}}
  \delta(x\cos\psi+z\sin\psi)e^{iq_xx}e^{iq_zz} \nonumber\\
  &= \frac{1}{\cos\psi}\int_{-\frac{D}{2}}^{\frac{D}{2}}\dz \rhoh{i} \exp\braces{
    -\frac{(z-\zh{i}\cos\psi)^2}{2\sigmah{i}^2\cos^2\psi} + i(q_z-q_x\tan\psi)z
  } \nonumber \\
  &\approx \rhoh{i}\sqrt{2\pi}\sigmah{i} \,\mathrm{exp}
  \braces{
    i\alpha\zh{i} - \frac{1}{2}\alpha^2\sigmah{i}^2
  } \label{eq:gauss_FT}
\end{align}
with $\alpha=q_z\cos\psi-q_x\sin\psi$.
Using Eq.~(\ref{eq:gauss_FT}) and adding the other side of the bilayer and
the terminal methyl term, we get
\begin{multline}
  F_\mathrm{G} = \sqrt{2\pi}
  \Bigg[
    -\rhom\sigmam \exp\braces{
      -\frac{1}{2}\alpha^2\sigmam^2
    } \\
    + \sum_{i=1}^{1\text{ or }2}2\rhoh{i}\sigmah{i}
    \cos(\alpha\zh{i})
    \,\mathrm{exp}\braces{-\frac{1}{2}\alpha^2\sigmah{i}^2}
  \Bigg].
\end{multline}
The strip part of the 
model in the minus fluid convention is
\begin{equation}
  \rhos(z) = \left\{
    \begin{array}{ccc}
      -\Delta\rho & \text{for } & 0 \leq z < \zchtwo\cos\psi, \\
      0   & \text{for } & \zw\cos\psi \leq z \leq D/2,
    \end{array}
  \right.
\end{equation}
where $\Delta\rho=\rhow-\rhochtwo$.
Then, the corresponding Fourier transform is 
\begin{align}
  F_\mathrm{S} 
  &= \iint\dz\dx e^{iq_xx}e^{iq_zz} \rhos(z)\delta(x\cos\psi+z\sin\psi) \nonumber\\
  &= \frac{2}{\cos\psi} \int_0^{\zchtwo\cos\psi}\dz\cos\pars{\frac{\alpha}{\cos\psi} z}(-\Delta\rho) \nonumber\\
  &= -2\Delta\rho\frac{\sin(\alpha\zchtwo)}{\alpha}.
\end{align} 
The bridging part of the model in the minus fluid convention is 
\begin{align}
  \rhob(x,z) = \frac{\Delta\rho}{2} \cos \bracks{
    \frac{-\pi}{\deltazh}(z'-\zw)} - \frac{\Delta\rho}{2}
\end{align}
for $\zchtwo\cos\psi < z < \zw\cos\psi$, and 0 otherwise. Here,
$\deltazh=\zw-\zchtwo$.
Then, for the strip part of the form factor, we have
\begin{align}
  F_\mathrm{B} 
  &= \iint\dz\dx e^{iq_xx}e^{iq_zz} \delta(x\cos\psi+z\sin\psi) \rhob(x,z) \nonumber\\
  &= \frac{\Delta\rho}{\cos\psi}
     \int_{\zchtwo\cos\psi}^{\zw\cos\psi}\dz \cos\pars{\alpha\frac{z}{\cos\psi}} 
     \braces{\cos\bracks{-\frac{\pi}{\deltazh}\left(\frac{z}{\cos\psi}-\zw\right)} - 1} \nonumber\\
  &= \Delta\rho \braces{
       \frac{\deltazh\sin\bracks{\frac{\pi(-u+\zw)}{\deltazh}+\alpha u}}{-2\pi+2\alpha\deltazh}
       + \frac{\deltazh\sin\bracks{\frac{\pi(u-\zw)}{\deltazh}+\alpha u}}{2\pi+2\alpha\deltazh}
       - \frac{\sin(\alpha u)}{\alpha}  
     }\Bigg|_{\zchtwo}^{\zw} \nonumber\\
  &= -\frac{\Delta\rho}{\alpha}\bracks{\sin(\alpha\zw)-\sin(\alpha\zchtwo)} \nonumber\\
  & \,\quad + \frac{\Delta\rho}{2} \pars{
      \frac{1}{\alpha+\frac{\pi}{\deltazh}} 
      + \frac{1}{\alpha-\frac{\pi}{\deltazh}}
    }\bracks{\sin(\alpha\zw)+\sin(\alpha\zchtwo)}.
\end{align}
Because our X-ray scattering intensity was measured in a relative scale, 
an overall scaling factor was necessary for a non linear least square 
fitting procedure. This means that $\Delta\rho$ can be absorbed in the 
scaling factor. Doing so means that the values of $\rhoh{i}$ and $\rhom$
resulting from a fitting procedure are relative to $\Delta\rho$. One way 
to have these parameters in the absolute scale is to integrate the 
bilayer electron density over the lipid volume and equate the result
to the total number of electrons in the lipid, which can easily be calculated
from the chemical formula. For the ripple phase study in this thesis, the
absolute values of the electron density were not of importance, so the
discussion was omitted in the main text.

%%%%%%%%%%%%%%%%%%%%%%%%%%%%%%%%%%%%%%%%%%%%%%%%%%%%%%%%%%%%%%%%%%%%%%%%%%%%%%%
\section{Correction due to refractive index}
$q_z$ needs be corrected for index of refraction. This section is practically
the same as an appendix in Yufeng Liu's thesis. I include this section for
mere convenience.

Let $\theta'$ and $\lambda'$ be the true scattering angle and wavelength
within the sample. The wavelength by an energy analyzer, $\lambda$, and the 
scattering angle calculated from a position on a CCD detector, $\theta$ are 
apparent. The correction is not necessary in the horizontal direction.
The Snell's law in Fig. X gives
\begin{align}
  n\cos\theta &= n'\cos\theta' \\
  n\lambda &= n'\lambda'.
\end{align}
For low angle X-ray scattering, the momentum transfer along $z$ direction is
\begin{align}
  q_z &= \frac{4\pi\sin\theta'}{\lambda'} \\
      &= \frac{4\pi n'}{n\lambda}\sin\theta' \\
      &= \frac{4\pi n'}{n\lambda}\sqrt{1-\cos^2\theta'} \\
      &= \frac{4\pi n'}{n\lambda}\sqrt{1-\left(\frac{n}{n'}\cos\theta\right)^2}.
\end{align}
The apparent scattering angle, $\theta$, is directly related to the vertical
pixel position, $p_z$, by 
\begin{equation}
  \theta = \frac{1}{2}\tan^{-1}\left(\frac{p_z}{S}\right),
\end{equation}
where $S$ is the sample-to-detector distance. The typical units of $S$ and 
$p_z$ are in mm. In our experimental setup,
$n=1$ and $n'=0.9999978$ for lipids at $\lambda=1.18$ \AA. 
