\chapter{Tat}
\section{Fixed Angle Analysis of NFIT}

\section{Domain Size Distribution: Gaussian and Exponential}

\section{Hard Wall Constraints on SDP}

\section{Nonsymmetrized Profiles of MD}

\chapter{Ripple Phase}
\section{Rotation of a Two-Dimensional Function}
Let us consider rotating a function, $f(x,z)$ in two dimensions by an angle, 
$\psi$, in the counterclockwise direction (see Fig. X). This is easily 
achieved by rotating the coordinate system by $\psi$ in the clockwise direction. 
Let rotated coordinates be $x'$ and $z'$. A point in the original coodinates,
($x$, $z$), is written as ($x'$, $z'$) in the new coordinates. More specifically,
the point P is written as 
$\mathbf{P}=x\xhat+z\zhat=x'\xhat'+z'\zhat'$. $\xhat$ and $\zhat$ in
the $x'z'$ coordinate system are written as 
\begin{align}
  \xhat &= \cos\psi\xhat'+\sin\psi\zhat' \\
  \zhat &= -\sin\psi\xhat'+\cos\psi\zhat'.
\end{align}
Pluggin these in $\mathbf{P}=x\xhat+z\zhat$ leads to
\begin{align}
  x' &= x\cos\psi - z\sin\psi \\
  z' &= z\cos\psi + x\sin\psi,
\end{align}
the inverse of which is
\begin{align}
  x &= x'\cos\psi + z'\sin\psi \\
  z &= -x'\sin\psi + z'\cos\psi.
\end{align}
Using the latter equations, $f(x,z)$ can be expressed in terms of $x'$ and $z'$. 
The resulting function $f(x',z')$ is the rotated version of $f(x,z)$. 

As an 
example, let us consider a Dirac delta function located at $(x,z)=(0,\zh)$,
that is, $f(x,z)=\delta(x)\delta(z-\zh)$. After the rotation by $\psi$, it 
becomes
\begin{align*}
  f(x,z) 
  &\rightarrow 
    \delta(x\cos\psi+z\sin\psi) \delta(-x\sin\psi+z\cos\psi-\zh) \\
  &= \frac{\delta(x+z\tan\psi)}{|\cos\psi|}
     \frac{\delta(-x\sin\psi\cos\psi+z\cos^2\psi-\zh\cos\psi)}{1/|\cos\psi|} \\
  &= \delta(x+z\tan\psi)\delta(z\tan\psi\sin\psi\cos\psi+z\cos^2\psi-\zh\cos\psi) \\
  &= \delta(x+z\tan\psi)\delta(z-\zh\cos\psi),
\end{align*}
which is a part of the expression for $T_\psi(x,z)$ in the simple delta 
function model.


%%%%%%%%%%%%%%%%%%%%%%%%%%%%%%%%%%%%%%%%%%%%%%%%%%%%%%%%%%%%%%%%%%%%%%%%%%%%%%%
\section{Derivation of the transbilayer part of the form factor in the 2G hybrid model}
In this section, we derive the trasbilayer part of the form factor calculated
from the 2G hybrid model discussed in section X.
Defining $z'=-x\sin\psi+z\cos\psi$, the Fourier transform of a Gaussian function 
along the line tilted from $z$-axis by $\psi$ is
\begin{align}
  & \iint\dz\dx \rhoh{i} \exp\braces{-\frac{(z'-\zh{i})^2}{2\sigmah{i}^2}}
  \delta(x\cos\psi+z\sin\psi)e^{iq_xx}e^{iq_zz} \nonumber\\
  &= \frac{1}{\cos\psi}\int_{-\frac{D}{2}}^{\frac{D}{2}}\dz \rhoh{i} \exp\braces{
    -\frac{(z-\zh{i}\cos\psi)^2}{2\sigmah{i}^2\cos^2\psi} + i(q_z-q_x\tan\psi)z
  } \nonumber \\
  &\approx \rhoh{i}\sqrt{2\pi}\sigmah{i} \,\mathrm{exp}
  \braces{
    i\alpha\zh{i} - \frac{1}{2}\alpha^2\sigmah{i}^2
  } \label{eq:gauss_FT}
\end{align}
with $\alpha=q_z\cos\psi-q_x\sin\psi$.
Using Eq.~(\ref{eq:gauss_FT}) and adding the other side of the bilayer and
the terminal methyl term, we get
\begin{multline}
  F_\mathrm{G} = \sqrt{2\pi}
  \Bigg[
    -\rhom\sigmam \exp\braces{
      -\frac{1}{2}\alpha^2\sigmam^2
    } \\
    + \sum_{i=1}^{1\text{ or }2}2\rhoh{i}\sigmah{i}
    \cos(\alpha\zh{i})
    \,\mathrm{exp}\braces{-\frac{1}{2}\alpha^2\sigmah{i}^2}
  \Bigg].
\end{multline}
The strip part of the 
model in the minus fluid convention is
\begin{equation}
  \rhos(z) = \left\{
    \begin{array}{ccc}
      -\Delta\rho & \text{for } & 0 \leq z < \zchtwo\cos\psi, \\
      0   & \text{for } & \zw\cos\psi \leq z \leq D/2,
    \end{array}
  \right.
\end{equation}
where $\Delta\rho=\rhow-\rhochtwo$.
Then, the corresponding Fourier transform is 
\begin{align}
  F_\mathrm{S} 
  &= \iint\dz\dx e^{iq_xx}e^{iq_zz} \rhos(z)\delta(x\cos\psi+z\sin\psi) \nonumber\\
  &= \frac{2}{\cos\psi} \int_0^{\zchtwo\cos\psi}\dz\cos\pars{\frac{\alpha}{\cos\psi} z}(-\Delta\rho) \nonumber\\
  &= -2\Delta\rho\frac{\sin(\alpha\zchtwo)}{\alpha}.
\end{align} 
The bridging part of the model in the minus fluid convention is 
\begin{align}
  \rhob(x,z) = \frac{\Delta\rho}{2} \cos \bracks{
    \frac{-\pi}{\deltazh}(z'-\zw)} - \frac{\Delta\rho}{2}
\end{align}
for $\zchtwo\cos\psi < z < \zw\cos\psi$, and 0 otherwise. Here,
$\deltazh=\zw-\zchtwo$.
Then, for the strip part of the form factor, we have
\begin{align}
  F_\mathrm{B} 
  &= \iint\dz\dx e^{iq_xx}e^{iq_zz} \delta(x\cos\psi+z\sin\psi) \rhob(x,z) \nonumber\\
  &= \frac{\Delta\rho}{\cos\psi}
     \int_{\zchtwo\cos\psi}^{\zw\cos\psi}\dz \cos\pars{\alpha\frac{z}{\cos\psi}} 
     \braces{\cos\bracks{-\frac{\pi}{\deltazh}\left(\frac{z}{\cos\psi}-\zw\right)} - 1} \nonumber\\
  &= \Delta\rho \braces{
       \frac{\deltazh\sin\bracks{\frac{\pi(-u+\zw)}{\deltazh}+\alpha u}}{-2\pi+2\alpha\deltazh}
       + \frac{\deltazh\sin\bracks{\frac{\pi(u-\zw)}{\deltazh}+\alpha u}}{2\pi+2\alpha\deltazh}
       - \frac{\sin(\alpha u)}{\alpha}  
     }\Bigg|_{\zchtwo}^{\zw} \nonumber\\
  &= -\frac{\Delta\rho}{\alpha}\bracks{\sin(\alpha\zw)-\sin(\alpha\zchtwo)} \nonumber\\
  & \,\quad + \frac{\Delta\rho}{2} \pars{
      \frac{1}{\alpha+\frac{\pi}{\deltazh}} 
      + \frac{1}{\alpha-\frac{\pi}{\deltazh}}
    }\bracks{\sin(\alpha\zw)+\sin(\alpha\zchtwo)}.
\end{align}
Because our X-ray scattering intensity was measured in a relative scale, 
an overall scaling factor was necessary for a non linear least square 
fitting procedure. This means that $\Delta\rho$ can be absorbed in the 
scaling factor. Doing so means that the values of $\rhoh{i}$ and $\rhom$
resulting from a fitting procedure are relative to $\Delta\rho$. One way 
to have these parameters in the absolute scale is to integrate the 
bilayer electron density over the lipid volume and equate the result
to the total number of electrons in the lipid, which can easily be calculated
from the chemical formula. For the ripple phase study in this thesis, the
absolute values of the electron density were not of importance, so the
discussion was omitted in the main text.


%%%%%%%%%%%%%%%%%%%%%%%%%%%%%%%%%%%%%%%%%%%%%%%%%%%%%%%%%%%%%%%%%%%%%%%%%%%%%%%
\section{Derivation of the contour part of the form factor}
In this section, we derive $\FC$. The ripple profile, $u(x)$ is given by
\begin{equation}
  u(x) = \left\{
    \begin{array}{ccc}
    -\frac{A}{\lambda_r-x_0}\left(x+\frac{\lambda_r}{2}\right) 
      & \text{for} 
      & -\frac{\lambda_r}{2} \leq x < -\frac{x_0}{2} \\
    \frac{A}{x_0}x 
      & \text{for} 
      & -\frac{x_0}{2} \leq x \leq \frac{x_0}{2} \\
    -\frac{A}{\lambda_r-x_0} \left(x-\frac{\lambda_r}{2}\right)
      & \text{for} 
      & \frac{x_0}{2} < x \leq \frac{\lambda_r}{2}
    \end{array} \right.
\end{equation}

The contour part of the form factor is the Fourier transform of the contour
function, $C(x,z)$,
\[
  \FC(\mathbf{q}) = \frac{1}{\lambda_r}
  \int_{-\frac{\lambda_r}{2}}^{\frac{\lambda_r}{2}}\dx
  \int_{-\frac{D}{2}}^\frac{D}{2}\dz 
  C(x,z) e^{iq_zz} e^{iq_xx}
\] 
As discussed in section X, the modulated models allow
the electron density to modulate along the ripple direction, $x$. This means
\begin{align}
  C(x,z) &= \left\{
  \begin{array}{ccc}
    f_1\delta[z-u(x)] & \text{for} & -\frac{\lambda_r}{2} \leq x < -\frac{x_0}{2} \\
    \delta[z-u(x)] & \text{for} & -\frac{x_0}{2} < x < \frac{x_0}{2} \\
    f_1\delta[z-u(x)] & \text{for} & \frac{x_0}{2} \leq x < \frac{\lambda_r}{2} \\    
  \end{array}
  \right. \nonumber\\
  &+ f_2\,\delta\!\pars{x+\frac{x_0}{2}}\delta\!\pars{z+\frac{A}{2}} 
   + f_2\,\delta\!\pars{x-\frac{x_0}{2}}\delta\!\pars{z-\frac{A}{2}}.
\end{align}
The contribution from the minor arm is
\begin{align}
  & \frac{1}{\lambda_r}
  \int_{-\frac{\lambda_r}{2}}^{-\frac{x_0}{2}}\dx e^{iq_xx} e^{iq_zu(x)}
  + \int_{\frac{x_0}{2}}^{\frac{\lambda_r}{2}}\dx e^{iq_xx} e^{iq_zu(x)} \nonumber\\
  &= \frac{1}{\lambda_r}
     \int_{\frac{x_0}{2}}^{\frac{\lambda_r}{2}}\dx 
     e^{-i\left[q_xx-q_z\frac{A}{\lambda_r-x_0}\left(x-\frac{\lambda_r}{2}\right)\right]}
     + \int_{\frac{x_0}{2}}^{\frac{\lambda_r}{2}}\dx 
     e^{i\left[q_xx-q_z\frac{A}{\lambda_r-x_0}\left(x-\frac{\lambda_r}{2}\right)\right]} \nonumber\\
  &= \frac{2}{\lambda_r}
     \int_{\frac{x_0}{2}}^{\frac{\lambda_r}{2}}   
     \cos\bracks{\pars{q_x-q_z\frac{A}{\lambda_r-x_0}}x
                 +q_z\frac{A}{\lambda_r-x_0}\frac{\lambda_r}{2}} \label{eq:minor_arm1}
\end{align}
Using a trigonometric identity, 
\[
  \sin u-\sin v = 2\cos[(u+v)/2]\sin[(u-v)/2],
\]
and defining 
\begin{equation}
  \omega(\mathbf{q}) = \frac{1}{2}\left(q_xx_0 + q_zA\right),
\end{equation}
we further simplify Eq.~(\ref{eq:minor_arm1}),
\begin{align}
  &= \frac{2}{\lambda_r}\frac{\lambda_r-x_0}{\frac{1}{2}q_x\lambda_r - \omega} 
     \cos\bracks{\frac{1}{2}\left(\frac{1}{2}q_x\lambda_r + \omega\right)} 
     \sin\bracks{\frac{1}{2}\left(\frac{1}{2}q_x\lambda_r - \omega\right)} \nonumber\\
  &= \frac{1}{\lambda_r}\frac{\lambda_r-x_0}{\frac{1}{2}q_x\lambda_r - \omega} 
     \cos\bracks{\frac{1}{2}\left(\frac{1}{2}q_x\lambda_r + \omega\right)} 
     \frac{\sin\left(\frac{1}{2}q_x\lambda_r - \omega \right)}
          {\cos\bracks{\frac{1}{2}\left(\frac{1}{2}q_x\lambda_r - \omega\right)}} \nonumber\\
  &= \frac{\lambda_r-x_0}{\lambda_r}
     \frac{\cos\bracks{\frac{1}{2}\left(\frac{1}{2}q_x\lambda_r + \omega\right)}}
          {\cos\bracks{\frac{1}{2}\left(\frac{1}{2}q_x\lambda_r - \omega\right)}}
     \frac{\sin\left(\frac{1}{2}q_x\lambda_r - \omega\right)}
          {\frac{1}{2}q_x\lambda_r - \omega}.
\end{align}
Similarly, we calculate the contribution from the major arm,
\begin{align}
  \frac{1}{\lambda_r}\int_{-\frac{x_0}{2}}^{\frac{x_0}{2}}\dx 
  e^{i\left(\frac{q_zA}{x_0} + q_x \right)x}
  &= \frac{2}{\lambda_r}\int_{0}^{\frac{x_0}{2}}\dx \cos\left(\frac{q_zA}{x_0} + q_x\right)x \nonumber\\ 
  &= \frac{x_0}{\lambda_r}\frac{\sin\omega}{\omega}
\end{align}
The contribution from the kink region is 
\begin{align}
  & \frac{1}{\lambda_r}\iint\dx\dz
  \bracks{\delta\!\pars{x+\frac{x_0}{2}}\delta\!\pars{z+\frac{A}{2}} 
   + \delta\!\pars{x-\frac{x_0}{2}}\delta\!\pars{z-\frac{A}{2}}}
  e^{iq_xx} e^{iq_zz} \nonumber\\
  &= \frac{2}{\lambda_r}\cos\omega.
\end{align}
Therefore,
\begin{align}
  \FC(\mathbf{q}) 
  &= \frac{x_0}{\lambda_r}\frac{\sin\omega}{\omega} + 
  f_1\frac{\lambda_r-x_0}{\lambda_r}
  \frac{\cos\bracks{\frac{1}{2}\left(\frac{1}{2}q_x\lambda_r + \omega\right)}}
       {\cos\bracks{\frac{1}{2}\left(\frac{1}{2}q_x\lambda_r - \omega\right)}}
  \frac{\sin\left(\frac{1}{2}q_x\lambda_r - \omega\right)}
       {\frac{1}{2}q_x\lambda_r - \omega} \nonumber\\
  &+ \frac{2f_2}{\lambda_r}\cos\omega
\end{align}


%%%%%%%%%%%%%%%%%%%%%%%%%%%%%%%%%%%%%%%%%%%%%%%%%%%%%%%%%%%%%%%%%%%%%%%%%%%%%%%
\section{Correction due to refractive index}
$q_z$ needs be corrected for index of refraction.
Let $\theta'$ and $\lambda'$ be the true scattering angle and wavelength
within the sample. The wavelength by an energy analyzer, $\lambda$, and the 
scattering angle calculated from a position on a CCD detector, $\theta$ are 
apparent. The correction is not necessary in the horizontal direction.
The Snell's law in Fig. X gives
\begin{align}
  n\cos\theta &= n'\cos\theta' \\
  n\lambda &= n'\lambda'.
\end{align}
For low angle X-ray scattering, the momentum transfer along $z$ direction is
\begin{align}
  q_z &= \frac{4\pi\sin\theta'}{\lambda'} \\
      &= \frac{4\pi n'}{n\lambda}\sin\theta' \\
      &= \frac{4\pi n'}{n\lambda}\sqrt{1-\cos^2\theta'} \\
      &= \frac{4\pi n'}{n\lambda}\sqrt{1-\left(\frac{n}{n'}\cos\theta\right)^2}.
\end{align}
The apparent scattering angle, $\theta$, is directly related to the vertical
pixel position, $p_z$, by 
\begin{equation}
  \theta = \frac{1}{2}\tan^{-1}\left(\frac{p_z}{S}\right),
\end{equation}
where $S$ is the sample-to-detector distance. The typical units of $S$ and 
$p_z$ are in mm. In our experimental setup,
$n=1$ and $n'=0.9999978$ for lipids at $\lambda=1.18$ \AA. 
