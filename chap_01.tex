\chapter{Introduction}
Biomembranes are blah blah blah. Various thermodynamic phases exist (see fig).
In this thesis, we forcus on the fluid and ripple phase. In the former phase,
we investigated the interaction of a peptide called Tat with lipid bilayers
in the fluid phase. The overview on this field is discussed in section 1.1.
The results and interpretation are discussed in chapter 3 and 4. Regarding
the ripple phase, we measured the electron density profile of the lipid
bilayers using a stack of oriented bilayers. Using wide angle x-ray scattering
technique, we also investigated the chain packing within a bilayer. The 
overview is discussed in section 1.2. The results and interpretation are 
discussed in chapter 5 and 6. The appendices show a lot of details that will
allow other people to reproduce much of the results shown in this thesis 
as well as help readers understand scattering analysis employed in this work.
It is the author's hope that these details will help future researchers,
especially students, understand some of the techniques to investigate the
structure of lipid bilayers in sub Angstrom resolution.

\section{HIV life cycle}
Basically, this section will introduce what HIV does to human cells in order
to put Tat in a context. Follow some of the review papers on HIV.

\section{Overview of Tat}
The name cell-penetrating peptide (CPP) connotes a peptide that 
easily penetrates cell membranes (for Reviews see [1-3]). 

This thesis focuses on 
the transactivator of translation, Tat, from the HIV-1 virus, which plays a 
role in AIDS progression. Earlier work showed that the HIV-Tat 
protein (86 amino acids) was efficiently taken up by cells, and concentrations 
as low as 1 nM were sufficient to transactivate a reporter gene expressed from 
the HIV-1 promoter [4, 5]. It has been reported that Tat protein uptake does not 
require ATP [6]. Studies using inhibitors of different types of endocytosis, 
including clathrinand caveolae-mediated, or receptor-independent 
macropinocytosis reached the same conclusion that ATP mediated endocytosis is 
not involved in Tat protein permeation [7-10]. However, this issue is 
controversial, as other studies found evidence for endocytosis in Tat protein 
import [11-19]. Still other studies have concluded that an ATP requirement for 
Tat protein entry depends on the size of the cargo attached to Tat protein, or 
on the specific cell type [20-22]. The part of the Tat protein responsible for 
cellular uptake was assigned to a short region Tat (48-60), G48RKKRRQRRRPPQ60, 
which is particularly rich in basic amino acids [6]. Deletion of three out of 
eight positive charges in this region caused loss of its ability to translocate 
[6]. In this manuscript short basic regions will be called Tat, while the 
entire 86-
amino acid protein will be called Tat protein. Tat was shown to be responsible 
for the Tat
protein’s permeation into the cell nucleus and the nucleoli [6], and this was 
confirmed using live
cell fluorescence in SVGA cells [23]. Tat (48-60) was shown to have little 
toxicity on HeLa
cells at 100 μM concentration [6], but the longer Tat protein (2-86) was toxic 
to rat brain glioma
cells at 1-10 μM [24]. Interestingly, no hemolytic activity was found when 
human erythrocytes
were incubated with a highly neurotoxic concentration (40 μM) of Tat (2-86) 
[24]. These results
prompt the question, what is the mechanism of Tat’s translocation through 
membranes?
To address this question, many biophysical studies have used simple models of
biological membranes composed of a small number of lipid types. These studies 
are valuable
because there is no possibility for ATP-dependent translocation, thus ruling 
out endocytosis if
translocation occurs. For example, Mishra et al. reported that the rate of 
entry into giant
unilamellar vesicles (GUVs) composed of PS/PC (1:4 mole ratio) lipids of 
rhodamine-tagged Tat
is immeasurably slow, but it crosses a GUV composed of PS/PC/PE (1:2:1) lipids 
within 30
seconds [25]. This study suggests that negative curvature induced by the 
inclusion of PE
facilitates translocation. In a subsequent study using much smaller unilamellar 
vesicles (LUVs),
Tat did not release an encapsulated fluorescent probe in LUVs composed of 
lipids modeling the
outer plasma membrane, PC/PE/SM/Chol (1:1:1:1.5), but did release the probe in 
LUVs
composed of BMP/PC/PE (77:19:4) [26]; BMP (bis(monoacylglycero)-phosphate) is 
an anionic
lipid specific to late endosomes. In that study [26], the inclusion of PE did 
not suffice to cause
leaky fusion in LUVs in the absence of a negatively charged lipid. The 
contrasting results in
these two experiments may also be due to the use of LUVs instead of GUVs since 
it was reported
that Tat does not translocate across LUVS of PC/PG (3:2) but does translocate 
across GUVs of
the same lipid composition [27]. In a similar experiment, Tat did not 
translocate into egg PC
LUVs [28]. In another experiment confirming these results, Tat did not 
translocate into GUVS
containing only PC with 20 mol% cholesterol, but when PS or PE was included 
with PC, then
rapid translocation of Tat was observed [29]. These experiments demonstrate 
that the choice of
lipids and model systems influences Tat translocation.

Is a pore formed during Tat translocation? Although direct conductance 
measurements of
Tat and lipid membranes have not been carried out, two studies measured 
conductance with the
somewhat similar CPP oligoarginine R9C peptide. Using single-channel 
conductance of
gramicidin A in planar lipid membranes consisting of anionic, neutral or 
positively charged
lipids, R9C did not increase conductance, even in anionic lipid membranes [30]. 
By contrast, in
a similar experiment using planar lipid membranes, a current was induced by R9C 
in PC/PG
(3:1) membranes, with increasing destabilization over time [31]. Thus questions 
remain about
pore formation of Tat in membranes. In the GUV experiment with Tat mentioned 
above [29],
Ciobanasu et al., using size exclusion methods, suggested a pore in the 
nanometer range, which
could only be passed by small dye tracer molecules. Thus, if a true pore forms, 
it is likely to be
small and transitory.

What is the secondary structure of Tat in membranes? Circular dichroism (CD)
spectroscopy was carried out on, where the penultimate proline on Tat (48-60) 
was replaced by a
tryptophan [27]. That study found a random coil secondary structure in aqueous 
solution as well
as when mixed with PC/PG/PE (65:35:5) LUVs. The same result was obtained using 
CD in
PC/PG (3:1) vesicles by Ziegler et al.[10], indicating that an alpha helix is 
not required for Tat’s
translocation ability. In addition, solid state NMR has identified a random 
coil structure of Tat in
DMPC/DMPG (8:7 mole ratio) multibilayers [32]. In the larger Tat-(1-72)-protein 
NMR
measurements at pH 4 have determined there is no secondary structure, with a 
dynamical basic
region [33]. Similarly, NMR was used to study the full Tat protein and found a 
highly flexible
basic region [34].

Regarding the mechanism of translocation of this randomly structured, short 
basic
peptide, many models have been proposed based on the conflicting results listed 
above.
Molecular dynamics simulations offer some insight into the molecular details of 
translocation.
Herce and Garcia simulated the translocation of Tat (Y47GRKKRRQRRR57) across 
DOPC at
various lipid:peptide molar ratios [35]. Their simulations indicated that Tat 
binds to the
phosphate headgroups, with 1 Tat binding with 14 lipids, each positive charge 
on Tat associated
with nearly 2 phosphate groups [35]. Translocation involved a localized 
thinning, and
snorkeling of arginine side chains through the hydrophobic layer to interact 
with phosphates on
the other side of the membrane. This allowed some water molecules to penetrate 
the membrane
along with Tat, forming a pore [35]. In this simulation, performed without 
inclusion of
counterions, pore formation was only observed at high ratios of peptide:lipid (1:18) 
or at
elevated temperature. However, a subsequent Gromacs simulation with counterions 
found no
thinning and no pore formation when Tat was added to DOPC membranes [36]. 
Instead it found
a membrane invagination associated with a cluster of Tat peptides, suggesting 
that
micropinocytosis could be the model for Tat translocation across membranes [36].
In this work we primarily combine experimental low-angle X-ray scattering (LAXS) 
data
with MD simulations to obtain the structure of fully hydrated, oriented lipid 
bilayers with Tat
(47-57) added at several mole ratios. The lipid systems were DOPC, DOPC/DOPE 
(3:1 mole
ratio), DOPC/DOPS (3:1), DOPC/DOPE (1:1) and a mimic of the nuclear membrane
(POPC/POPE/POPS/SoyPI/Chol, 69:15:2:4:11). Accessory techniques, densitometry, 
wideangle
X-ray scattering (WAXS), neutron scattering, CD spectroscopy were also applied 
to
further characterize Tat/membrane interactions.
