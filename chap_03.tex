\chapter{Structural Perturbation on Lipid Bilayer Due to Tat Peptide}
As discussed in chapter 2, the two main techniques employed in this thesis 
were molecular dynamics (MD) simulation and low angle X-ray scattering (LAXS).
First, we discuss the results and analysis of diffuse X-ray scattering. The
general protocol was the following; LAXS data were fitted to a model X-ray
scattering pattern from a stack of flucutating membranes via NFIT program,
the analysis of which yielded the bending modulus, $K_C$, and the bulk
modulus, $B$. Dividing the experimenal data by the model, then, gave the 
absolute X-ray form factor, $|F(q_z)$, which is the Fourier transform
of bilayer electron density profile along the bilayer normal direction, $z$. 
We fitted $|F(q_z)|$ to a model density profile using the scattering density
profile (SDP) program. The SDP program allows us to model a bilayer density
according to volumentric spacing constraint. The advantage of this program is
that we can see fine details of bilayer structure such as an individual
head group, terminal methly, and so on. The model requires many parameters
that are not so well determined. We then constrain many parameters from
the past experimental data and MD simulations. This is discussed in 
section ?.

The second main method is MD simulation. From simulation trajectory, we 
calculated the so called simulated X-ray form factor using the SIMtoEXP
program (ref). The best matching simulation result was chosen as the best
prediciton of the bilayer structure. We, then, calculated many structural
details from the trajectory that were not accessible experimentally.   

Section X discusses the implication of the results obtained in the proceeding 
sections. While this study does not probe dynamics of Tat translocation,
it supports Tat's ability to interact with neutral membranes. This finding
is compared with recent studies on a single arginine molecule.
 
\section{Low Angle X-ray Scattering}
\subsection{Theory}
The analysis of diffuse X-ray scattering pattern begins with separating 
$|F(q_z)|$ from $S(\mathbf{q})$. To this end, we used an analysis program
called NFIT. The derivation is described in Yufeng Liu's thesis in detail. 
In this section, we describe the theoretical model for $S(\mathbf{q})$ to 
outline the theory. 

We assume that a stack of bilayers can be accurately
described by the smectic liquid crystal theory, so that the free energy of 
the system is
\begin{equation}
  F
\end{equation}
where $K_C$ and $B$ are the bending and bulk modulus, respectively. 
Writing the membrane height profile in terms of the Fourier modes,
$u=\sum \mathrm{exp}(i\mathbf{q} \cdot \mathbf{r})$, 
\begin{equation}
  F
\end{equation}


\subsection{Results}
Fig. XX shows the scattering intensity pattern from DOPC/DOPE (1:1) with mole 
fraction
x=0.034 Tat. The diffuse lobes are due to equilibrium fluctuations that occur 
in these fully
hydrated, oriented lipid/peptide samples. The intensity I(q) in the diffuse 
patterns provide the
absolute values of the form factors F(qz), which are the Fourier transforms 
of the electron density
profile, through the relation I(q)=S(q)|F(qz)|2/qz, where q=(qr,qz), S(q) is 
the structure
interference factor, and qz
−1 is the usual LAXS approximation to the Lorentz factor [39, 55, 56].
The first step in the analysis takes advantage of the qr dependence of the 
scattering to obtain the
bending modulus KC with results shown in Fig. 2. As positively charged Tat 
concentration was
increased, the lamellar repeat spacing D generally increased in neutral lipid 
bilayers and
decreased in negatively charged bilayers, consistent with changes in 
electrostatic repulsive
interactions. With few exceptions, the water space between bilayers exceeded 
20 Å.
Figure 1. LAXS of DOPC/DOPE (1:1), x=0.034 Tat mole
fraction (peptide/(lipid+peptide)) at 37 oC. White lobes of
diffuse scattering intensity have large grey numbers, while
lamellar orders and beam are shown to the left of the
Molybdenum beam attenuator (short, dark rectangle). qz
and qr are the projections of q along the direction normal
and parallel to the membranes, respectively. The lamellar
repeat spacing was D=66.2 Å.
7
Figure 2. Bilayer bending modulus, KC, vs. P/(L+P) mole fraction. D-spacings 
for DOPC/Tat
mixtures varied from 64 to 68 Å, for DOPC/DOPE/Tat mixtures from 64 to 69 Å, for
DOPC/DOPS/Tat (3:1) mixtures from 57 Å to >100 Å (pure DOPS was unbound), and for
nuclear mimic/Tat mixtures from unbound (nuclear mimic) to 64 Å. Estimated uncertainty in all
values is ± 2.
The analysis that obtains KC also obtains the structure factor S(q) and then the unsigned
form factors |F(qz)| are obtained from the intensity I(q) by division. Results for five different
membrane mimics are shown in Fig. 3. Vertical lines indicate the “zero” position between the
lobes of diffuse data where F(qz) change sign. In every sample, the zero positions shift to larger
qz, indicating a thinning of the membranes.
8
Figure 3. Form factors of lipid mixtures (arbitrarily scaled and vertically displaced) with
increasing Tat mole fractions, P/(L+P), indicated on figure legends. Lipid mixtures: A. DOPC
B. DOPC/DOPE (3:1) C. DOPC/DOPE (1:1) D. DOPC/DOPS (3:1) E. Nuclear mimic. The
entire qz range is shown in C, while others show partial ranges. Solid vertical lines indicate the
qz values where the form factors equal zero between the lobes of diffuse data.

\section{Scattering Density Profile Modeling}
We also estimate structure by fitting the experimental form factors using the SDP method
[44] with the component groups identified in Fig. 5. The positions of these groups were free
parameters and the agreement with the experimental form factors was excellent. Absolute total
electron density profiles and the Tat profiles are shown for many samples in Fig. 6 (A-C). It
must be emphasized, however, that, while the total EDP is well determined by this fitting
procedure, the values of the parameters for the components are not as well determined as they
would be if one had X-ray data to smaller and larger qz and neutron data. Indeed, there are local
minima in the fitting landscape, including one with Tat closer to the center of the bilayer as
shown in Fig. S5. The simulations help to discard that result. For the results shown in Fig. 6, a
consistent trend is that Tat moves away from the bilayer center as concentration increases.
Electron density profiles for DOPC/DOPS (3:1) and the nuclear membrane mimic were not
successful, due to loss of diffuse scattering by Tat’s charge neutralization of these negatively
charged membranes.
11
Figure 6. SDP modeling results for
absolute electron density profiles
(EDPs) and for the Tat location as a
function of distance Z along the
bilayer normal. A. DOPC B.
DOPC/DOPE (3:1) and C.
DOPC/DOPE (1:1).
12
More structural detail from the modeling and from the simulations is shown in Fig. 7. The
bilayer thickness can be described as DHH, which is the distance between the maxima in the
electron density profile, or as DPP, which is the distance between the phosphocholines on the
opposing monolayers (see Fig. 5). Figs. 7A and 7B show that both these quantities tend to
decrease with increasing Tat mole fraction (P/(L+P)), showing that Tat thins membranes,
increasingly so as its concentration is increased, even though both simulation and modeling
suggest that Tat moves further from the membrane center with increasing concentration as
shown in Fig. 7D. Fig. 7C shows that the area per lipid AL usually increases with increasing
mole fraction of Tat, similar to the findings from MD simulations (Section 3.2), as would be
expected. The results from the simulation data plotted in Fig. 7 were obtained by using a
weighted average based on chi-square of the four best fits of the simulated form factors with the
experimental form factors.
Figure 7. A. Bilayer thickness, DPP; B. Bilayer thickness, DHH; C. Area/lipid, AL; D. Twice the
Tat location, 2ATat: all plotted vs. Tat mole fraction P/(L+P). Error bars are standard deviations
from imposing Tat Gaussian widths, σ = 2.5, 3.0 or 3.5 Å. Inverted blue triangles connected
with dotted line are results from MD simulations, averaging the best fits to the X-ray data for
each parameter, with standard deviations shown.

\section{Molecular Dynamics Simulations}
Due to the slow relaxation in lipid bilayers and limited accuracy of the force field, a good
agreement between experimental and MD simulation calculated form factors may be difficult to
reach. Consequently, we carried out several constrained simulations at AL and ZTat as described
in Materials and Methods. We then compared the simulated form factor F(qz) with the
experiment. The best match for DOPC/Tat (128:4) was found when the Tats were constrained at
9
18 Å away from the bilayer center (Fig. 4.A,B). The other best fit results were: DOPC AL = 70
Å2 and DOPC/Tat(128:2) AL = 72 Å2, ZTat = 18 Å. It clearly indicates that with increasing Tat
concentration, AL increases. The agreement worsened as Tat was constrained to be closer to the
center of the bilayer. When Tats were constrained at 5 Å away from the bilayer center, we
observed a spontaneous formation of water pores in the MD simulation. However, as shown in
Fig. 4.C the corresponding form factor calculated from MD simulations does not match well
with experiments.
Figure 4. MD simulated form factors (red solid lines in A and C) of Tat/(DOPC+Tat), x=0.030,
with Tat fixed at ZTat= 18 Å (panel A) and 5 Å (panel C) from the bilayer center compared to
experimental form factors (open circles) scaled vertically to provide the best fit to the
simulations. Corresponding snapshots are shown in Panels B and D in which the lipid chains are
represented as grey sticks on a white background, Tats are yellow, phosphate groups are red and
water is blue.
10
Figure 5. Symmetrized total electron density profile (EDP) from the simulation with the form
factor shown in Fig. 4A. Also shown are the component group contributions. Component groups
are labeled: PC, phosphocholine; CG, carbonyl-glycerol; CH2+CH, methylene and methine
hydrocarbon region; CH3, terminal methyl; Tat peptide distribution is shaded.


\section{Volume results}
Experimental and simulated volumes are given in Table 2. The simulated volume was
obtained using the volume app in the SIMtoEXP program. The experimental Tat volume was
calculated from the measured density assuming that the lipid volume was the same as with no
Tat. In general, there may be an interaction volume between the peptide and the lipid membrane
as we found previously for bacteriorhodopsin [57]. As lipid was present in excess to Tat, the
partial molecular volume of the lipid should be the same as with no Tat, so this way of
calculating includes all the interaction volume in VTat. Comparison of VTat in water with the
result for 5:1 Lipid:Tat suggests that the interaction volume may be negative, consistent with a
net attractive interaction with lipid. Understandably, values of VTat were unreliable for small
mole ratios of Tat:Lipid. Therefore we used simple additivity for those mimics not shown in
Table 2 for the volumes used in the SDP program. All volumes obtained from the Gromacs MD
simulations were somewhat smaller than the measured volumes, but it supports the Tat volume
being closer to 1822 Å3 than the outlying values obtained experimentally at small Tat
concentrations.
Table 2. Volume results at 37 oC
Tat in: Vlipid (Å3) Lipid:Tat VTat (Å3)
Water 1877
DOPC/DOPE (3:1) 1288 5:1 1822
DOPC 1314 39.6:1 676
DOPC/DOPS (3:1) 1298 39.6:1 2613
Simulations
DOPC 1283 128:2 1694
DOPC 1294 128:4 1699

\section{Summary of Results}
We summarize our results for how Tat affects the lipid bilayer in Fig. 9. The height of
Tat, HTat = 8.7 Å, was the full width at half maximum of the Tat electron density profiles
obtained from simulations and the cylindrical radius, RTat = 8.3 Å, was calculated to give the
measured volume. The Z distances from the center of the bilayer were derived from weighted
averages of four MD simulations of Tat:DOPC 2:128. The χ2 obtained by comparison to
experiment indicated that the best ZTat lay between the simulated values of 16 Å and 18 Å and
the best area/lipid AL lay between the simulated values of 72 Å2 and 74 Å2, so averages were
obtained from these four combinations of ZTat and AL, weighted inversely with their χ2. The
average positions, Z'Phos, of phosphates situated underneath the Tats were calculated by
averaging over the phosphates whose in-plane distance, R, from the center of Tat is smaller than
RTat. The simulation cell extended to 38Å, far enough to ensure that ZPhos for most of the lipids is
the same as for DOPC. Assuming a simple linear ramp in ZPhos, Fig. 9 then indicates a ring of
boundary lipids that extends twice are far in R as Tat itself. Although the guanidinium electron
density profile was broad (Fig. S8), indicating that some were pointing away from the bilayer
relative to the center of Tat, more were pointing towards the bilayer center as indicated in Fig. 9.
Numerical values are given in Table S1.

\section{Discussion}

\section{Conclusion}
