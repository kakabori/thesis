\chapter{Materials and Methods}
This chapter describes parts of experimental techniques that are common
in both Tat and ripple phase projects.
These common parts are also the key experimental components that make
experiments on oriented samples successful. Experimental and theoretical
methods that are specific to each project are described in each
respective chapter.

\section{X-ray optics}
High resolution setup should be described.

Low resolution setup should be described.

\section{Hydration Chamber}
Probably the most important of all.
The chamber is sealed very tightly. We used helium to replace air. Air 
scattering is strong. 
Mylar window, cast scatteing in wide angle region. Helium. Peltier used to 
condense water in and out of the sample.  

\section{Sample Preparation}
\subsection{Stock Solutions}
Synthesized lipids were purchased from Avanti Polar Lipids (Alabaster, AL) and 
used without further purification. Membrane mimics for Tat experiments 
were prepared by first 
dissolving lyophilized lipids in chloroform and then mixing these stock 
solutions to create the lipid compositions
DOPC, DOPC/DOPE (3:1), DOPC/DOPE (1:1), DOPC/DOPS (3:1) and nuclear membrane
mimic (POPC/POPE/POPS/SoyPI/Cholesterol, 69:15:2:4:11) (based on Ref. [37]). 
Peptide
(Y47GRKKRRQRRR57) was purchased in two separate lots from the Peptide Synthesis 
Facility
(University of Pittsburgh, Pittsburgh, PA); mass spectroscopy revealed greater
than 95\% 
purity. This Tat
peptide corresponds to residues (47-57) of the 86 residues in the Tat 
protein [6]. Tat was
dissolved in HPLC trifluoroethanol (TFE) and then mixed with lipid stock 
solutions in
chloroform to form mole fractions between 0.0044 and 0.108. Weight of Tat in 
these mole
fractions was corrected for protein content (the remainder being 8 
trifluoroacetate counter-ions
from the peptide synthesis). Solvents were removed by evaporation in the fume 
hood followed
by 2 hours in a vacuum chamber at room temperature.

\subsection{Thin Film Samples}

For Tat experiments, four mg dried lipid/peptide mixture was re-dissolved in HPLC chloroform/TFE 
(2:1 v:v)
for most of the lipid compositions. 
DOPC/DOPS (3:1) mixtures required
chloroform/HFP (1:1 v:v) in order to solubilize the negatively charged DOPS. 
200 $\mu$l of 4 mg
mixtures in solvents were plated onto silicon wafers (15x30x1 mm) via the rock 
and roll method
[38] to produce stacks of $\approx$1800 well-aligned bilayers; 
solvents were removed by 
evaporation in
the fume hood, followed by two hours under vacuum. Samples were prehydrated 
through the
vapor in polypropylene hydration chambers at 37 \degC for two to six hours 
directly before hydrating in the
thick-walled X-ray hydration chamber [39] for 0.5 to 1 hour. 

For ripple phase experiments, four mg DMPC powder was dissolved in 
140 $\mu$l chloroform/methanol (2:1 v:v) mixture. The solution was
plated onto silicon wafers similarly to Tat mixtures. 
The sample was annealed at 60 \textcelsius\ for approximately
6-10 hours just before the X-ray experiment. 
Then, the sample was trimmed to 1 mm for high
resolution and 5 mm for low resolution study. The temperature
was set to 18 \textcelsius. 

\section{CCD detector}
Data reduction and correction for charged coupled device (CCD) detector
are discribed in detail in \cite{ref:Burner}.

\section{MD simulation}
Systems with different DOPC/Tat mole ratios (128:0, 128:2 and 128:4, corresponding to
0, 0.015 and 0.030 mole fractions) were simulated atomistically using the Gromacs 4.6.1
package [49]. DOPC was modeled by the Slipid force field [50, 51] and HIV Tat was modeled
by Amber 99SB [52]. Tip3p water was used [53]. The number of Tats was divided equally on
each side of the bilayer to mimic experimental conditions. All systems were simulated at 310 K
with a constant area in the x-y plane for and 1 atm constant pressure in the Z direction. Each
system was simulated for 100 ns and the last 50 ns was used as the production run.
At each DOPC/Tat mole ratio, we studied systems with three different area/lipid (AL).
For the DOPC system, we fixed AL = 68, 70, 72 Å2; DOPC/Tat (128:2), we fixed the AL = 72,
74, 76 Å2; DOPC/Tat (128:4), we fixed the AL =72, 74, 76 Å. For each DOPC/Tat system at
fixed AL, we then conducted seven independent simulations with the center of mass (COM) of
each Tat constrained at different bilayer depths from the bilayer center (18, 16, 14, 12, 10, 8 and
5 Å). In total, 45 independent simulations were conducted. The goal of constrained simulations
is to find the best match between experimental and MD simulation form factors. Comparison to
the X-ray form factors was performed using the SIMtoEXP software [54]. Additional details
concerning the MD simulations are in Supplementary Data 6.


