\chapter{Materials and Methods}
This chapter describes common parts of all the experiments in this thesis.
These common parts are also the key experimental components that makes
experiments on oriented samples successful. Experimental and theoretical
methods that are specific to each project are described in each
respective chapter.

\section{X-ray optics}
High resolution setup should be described.

Low resolution setup should be described.

Detector correction was applied. See references.

\section{Hydration Chamber}
The chamber is sealed very tightly. We used helium to replace air. Air 
scattering is strong. 
Mylar window, cast scatteing in wide angle region. Helium. Peltier used to 
condense water in and out of the sample.  

\section{Sample Preparation}
Synthesized lipids were purchased from Avanti Polar Lipids (Alabaster, AL) and 
used without further purification. Membrane mimics were prepared by first 
dissolving lyophilized lipids in chloroform and then mixing these stock 
solutions to create the lipid compositions
DOPC, DOPC/DOPE (3:1), DOPC/DOPE (1:1), DOPC/DOPS (3:1) and nuclear membrane
mimic (POPC/POPE/POPS/SoyPI/Cholesterol, 69:15:2:4:11) (based on Ref. [37]). 
Peptide
(Y47GRKKRRQRRR57) was purchased in two separate lots from the Peptide Synthesis 
Facility
(University of Pittsburgh, Pittsburgh, PA); mass spectroscopy revealed greater
than 95\% 
purity. This Tat
peptide corresponds to residues (47-57) of the 86 residues in the Tat 
protein [6]. Tat was
dissolved in HPLC trifluoroethanol (TFE) and then mixed with lipid stock 
solutions in
chloroform to form mole fractions between 0.0044 and 0.108. Weight of Tat in 
these mole
fractions was corrected for protein content (the remainder being 8 
trifluoroacetate counter-ions
from the peptide synthesis). Solvents were removed by evaporation in the fume 
hood followed
by 2 hours in a vacuum chamber at room temperature.

\section{Samples for X-ray scattering}
Four mg dried lipid/peptide mixture was re-dissolved in HPLC chloroform/TFE 
(2:1 v:v)
for most of the lipid compositions. DOPC/DOPS (3:1) mixtures required
chloroform/HFP (1:1 v:v) in order to solubilize the negatively charged DOPS. 
200 μl of 4 mg
mixtures in solvents were plated onto silicon wafers (15x30x1 mm) via the rock 
and roll method
[38] to produce stacks of ~1800 well-aligned bilayers; solvents were removed by 
evaporation in
the fume hood, followed by two hours under vacuum. Samples were prehydrated 
through the
vapor in polypropylene hydration chambers at 37 \degC for two to six hours 
directly before hydrating in the
thick-walled X-ray hydration chamber [39] for 0.5–1 h. Pre-equilibration 
allowed 
sufficient time
for equilibrium binding of peptides with membrane mimics.

For ripple phase experiments, the sample was annealed at 60 \degC for approximately
10 hours prior to the experiment. The sample was trimmed to 1 mm for high
resolution and 5 mm for low resolution study, respectively. The temperature
was set at 18 \degC. The sample was dried by applying a small heating 
current to the Peltier.

\section{CCD detector}
Data reduction and correction for charged coupled device (CCD) detector
are discribed in detail in \cite{ref:Burner}.

\section{Low Angle X-ray Scattering Experiment}
The analysis of diffuse low angle X-ray scattering from oriented stacks of 
fluctuating fluid bilayers has been previously described [39]. Absolute form 
factors |F(qz)| were obtained as previously described [42]. Modeling to 
estimate the locations of Tat and the lipid components
was performed using the SDP program [44].

\section{Transmission X-ray Scattering Experiment}
Transmission experiment. The axis of rotation does not coincide with the
plane of the sample, so that the sample-to-detector distance is not fixed
for different motor angle. The sample-to-detector distance was estimated
from the setup geometry. We used 35 $\mu$m thin Si wafer, which absorbs
x-ray by only 10\%. 

\section{Densimetry}
\subsection{Samples for densimetry}
Multilamellar vesicles (MLVs) were prepared by mixing dried lipid mixtures with 
MilliQ
water to a final concentration of 2-5 wt\% in nalgene vials and cycling three 
times between -20oC
and 60oC for ten minutes at each temperature with vortexing. Pure Tat was 
dissolved in water
at 0.4 wt\%.

\subsection{Densimetry}
Volumes of lipid mixtures with and without peptides in fully hydrated multilamellar
vesicles (MLV) were determined at 37±0.01 \degC using an Anton-Paar USA DMA5000M
(Ashland, VA) vibrating tube densimeter [47].


\section{MD simulation}
Systems with different DOPC/Tat mole ratios (128:0, 128:2 and 128:4, corresponding to
0, 0.015 and 0.030 mole fractions) were simulated atomistically using the Gromacs 4.6.1
package [49]. DOPC was modeled by the Slipid force field [50, 51] and HIV Tat was modeled
by Amber 99SB [52]. Tip3p water was used [53]. The number of Tats was divided equally on
each side of the bilayer to mimic experimental conditions. All systems were simulated at 310 K
with a constant area in the x-y plane for and 1 atm constant pressure in the Z direction. Each
system was simulated for 100 ns and the last 50 ns was used as the production run.
At each DOPC/Tat mole ratio, we studied systems with three different area/lipid (AL).
For the DOPC system, we fixed AL = 68, 70, 72 Å2; DOPC/Tat (128:2), we fixed the AL = 72,
74, 76 Å2; DOPC/Tat (128:4), we fixed the AL =72, 74, 76 Å. For each DOPC/Tat system at
fixed AL, we then conducted seven independent simulations with the center of mass (COM) of
each Tat constrained at different bilayer depths from the bilayer center (18, 16, 14, 12, 10, 8 and
5 Å). In total, 45 independent simulations were conducted. The goal of constrained simulations
is to find the best match between experimental and MD simulation form factors. Comparison to
the X-ray form factors was performed using the SIMtoEXP software [54]. Additional details
concerning the MD simulations are in Supplementary Data 6.


